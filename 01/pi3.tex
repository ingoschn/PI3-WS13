%\documentclass[Uebungsblatt]{pi3}
%\documentclass[Loesungsblatt]{pi3}
\documentclass{pi3}

%\Blatt{1}{ausgabe}{abgabe} % use with Uebungsblatt
\blatt{1} % use with Abgabe and Loesungsblatt
\tutor{tutor}
\uebungsgruppe{gruppe}
\teilnehmer{teilnehmer 1 \\ teilnehmer 2 \\ teilnehmer 2}

\begin{document}
\maketitle

\section{Titel der ersten Aufgabe}
Hier steht eine tolle Dokumentation\dots
\begin{code}
fak :: Integer -> Integer
fak 0 = 1
fak n = n * fak (n-1)
\end{code}

\begin{xcode}
data Tag = Montag | Dienstag | Mittwoch | Donnerstag | Freitag | Samstag | Sonntag
             deriving (Show, Eq, Ord, Ix, Enum)
\end{xcode}

%from file
Und hier aus einer Datei
\haskellcode{test}

%inline
Und hier dann auch inline
\hs{map toUpper :: [Char] -> [Char]}

\section{Titel der zweiten Aufgabe}
\begin{theorem}
Hier postuliert man sein Theorem...
\end{theorem}

\begin{proof}
Und hier wird es bewiessen...
\end{proof}
\end{document}